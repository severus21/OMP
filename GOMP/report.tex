\documentclass[a4paper,10pt]{article}

\usepackage[utf8]{inputenc} 
\usepackage[T1]{fontenc}     
\usepackage[francais]{babel}

\usepackage[toc,page]{appendix}

\usepackage{lmodern} % Pour changer le pack de police
\usepackage{makeidx}

\usepackage{color}
\usepackage[top=2cm, bottom=2cm, left=2cm, right=2cm]{geometry}




\usepackage{url}


\title{The Global Open Memory Project}
\author{\textsc{Prosperi} Laurent}
\date\today % Pour mettre la date du jour, tapez \today 
\makeindex

\begin{document}

\maketitle


\begin{abstract}
   The idea of this project is to save space and increase relaibility of storage to maintain open data. First, let's define 
   open data, this is a base of knowledge available for every human being, free of right. Note that  personnal data do not fit 
   into open data.
   \\
   
   Mainly, open data are now host by various bunches of instituts, organizations and so far are hosted by as many storage systems. 
   The improvement of the project is to build a global, distributed system to store all this data and to provide fault-tolerance, deduplication at 
   world scale.
\end{abstract}

\section{The objectives}
\begin{itemize}
 \item scalability
 \item fault-tolerance
 \item deduplication
 \item distributed
 \item performances
 \item safety
 \item code correctness
\end{itemize}



\end{document}
