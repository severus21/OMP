\section{The Object Naming Protocol(Convention ?)}

\textbf{se renseigner sur les différents systèmes de gestions de droits}

In this section we will discuss how to name( assign an identifier) to an object, can be a chunk or a file.

Let D : the data of the object
The Uniq Identifier Label :\\

schéma nécessaire

\begin{tabular}{c|c|p{0.7\textwidth}|}
4 Bytes & Command & 
\begin{enumerate}
 \item 1 byte : number of metadata indirection( nom un peu moche)
 \begin{itemize}
  \item For a default chunk : 0
  \item Otherwith see(le truc qui parle des métadonnées)
 \end{itemize}

\end{enumerate}\\
32 Bytes & Data hash & SHA256(D)\\

32/64 Bytes & Permission & ???????????????????????????????The encrypt version of the \textit{Data hash} je pense que c'est une très mauvaise idée à voir\\
\end{tabular}
\textbf{Remarque :}
Lets discuss about check sum, we will use CRC32(not 64) because it will be use on chunks, 
with a size less than ?(8KB).
Size overhead 10\%, is that worth it??????????????????? use to check hardware error, network 
error and not malicious action. But faster than sha256.\\\\
\textbf{Nota bene :}
We ought to ensure that $p(n,d) \leq 10^{-18}$, hence we can use SHA256 because storage system 
upper bound is about $2^{18}$Bytes.

\begin{proof}
The birthday problem can be generalized as follows given n random integers drawn from a 
discrete uniform distribution with range [1,d], what is the probability p(n;d) that at 
least two numbers are the same? (d=365 gives the usual birthday problem.)\\
   $p(n;d)\approx 1-e^{-n(n-1)/(2d)}$\\
   $p(n;d)\approx 1-\left({\frac {d-1}{d}}\right)^{n(n-1)/2}$\\\\
   \begin{table}[h]
      \centering
      \begin{tabular}{|c|c|c|c|c|c|c|c|c|}
	\hline
	Bits & Possible outputs &  \multicolumn{7}{c|}{Desired probability of random collision} \\
	\hline
	    & & $10^{-18}$ & $10^{-15}$ & $10^{-12}$ & $10^{-9}$ & $10^{-6}$ & 0.1\% & 1\% \\
	\hline
	16 & 65536  & $\le2$& $\le2$& $\le2$& $\le2$& $\le2$& 11 & 36  \\
	32 & $4.3*10^{9}$ & $\le2$& $\le2$& $\le2$& 3 & 93 & 2900 & 9300 \\
	64 & $1.8*10^{19}$ & 6 & 190 & 6100 & 190000 & 6100000 & $1.9*10^{8}$ & $6.1*10^{8}$ \\
	128 & $3.4*10^{38}$ & $2.6*10^{10}$ & $8.2*10^{11}$ & $2.6*10^{13}$ & $8.2*10^{14}$ & $2.6*10^{16}$ & $8.3*10^{17}$ & $2.6*10^{18}$\\
	256 & $1.2*10^{77}$ & $4.8*10^{29}$ & $1.5*10^{31}$ & $4.8*10^{32}$ & $1.5*10^{34}$ & $4.8*10^{35}$ & $1.5*10^{37}$ & $4.8*10^{37}$\\
	384 & $3.9*10^{115}$ & $8.9*10^{48}$ & $2.8*10^{50}$ & $8.9*10^{51}$ & $2.8*10^{53}$ & $8.9*10^{54}$ & $2.8*10^{56}$ & $8.9*10^{56}$\\
	512 & $1.3*10^{154}$ & $1.6*10^{68}$ & $5.2*10^{69}$ & $1.6*10^{71}$ & $5.2*10^{72}$ & $1.6*10^{74}$ & $5.2*10^{75}$ & $1.6*10^{76}$\\
	\hline
    \end{tabular}
    \caption{}
   \end{table}
   
ref:    https://fr.wikipedia.org/wiki/Paradoxe\_des\_anniversaires
  
\end{proof}


\textbf{Problem}
The permission handler as only one mod ie acces(write/read) or nothing. Other with data permission must be ensure by the 
metdata owner.