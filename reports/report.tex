\documentclass[a4paper,10pt]{article}

\usepackage[utf8]{inputenc} 
\usepackage[T1]{fontenc}     
\usepackage[francais]{babel}

\usepackage[toc,page]{appendix}	

\usepackage{lmodern} % Pour changer le pack de police
\usepackage{makeidx}

\usepackage{color}
\usepackage[top=2cm, bottom=2cm, left=2cm, right=2cm]{geometry}
\usepackage{url}
\usepackage{enumerate}

\usepackage{algorithmicx}
\usepackage{algorithm}
\usepackage{algpseudocode}	
 
\usepackage{amsthm}
\usepackage{amsmath}
\usepackage{amssymb} 	
\usepackage{stmaryrd}	

% \providecommand{\lxor}{\veebar}

\usepackage{multido}

\usepackage{caption}
\usepackage{tikz}


\usetikzlibrary{shapes,positioning}
\tikzset{node/.style={draw,circle,minimum size=1.2cm},}
% \tikzset{point/.style={draw,circle,minimum size=0cm},}


\newtheorem{theorem}{Theorem}[section]
\newtheorem{prop}{Proposition}[section]
\newtheorem{lemma}{Lemma} [section]
\newtheorem{remarque}{Remarque}[section]
\newtheorem{definition}{Définition}[section]

\title{The Hypercube Placement Algorithm}
\author{\textsc{Prosperi} Laurent}
\date\today % Pour mettre la date du jour, tapez \today 
\makeindex

\begin{document}

\maketitle


\begin{abstract}
   This work address strategy of data placement in ditributed, decentralized network, in order to increase reliability, 
   fault-tolerance and scalability.
   
\end{abstract}

\section{The objectives}
\begin{itemize}
 \item scalability
 \item fault-tolerance
 \item deduplication
 \item distributed
 \item performances
 \item safety( data integrity, ...)
 \item code correctness
\end{itemize}


\section{Quick Overview}
First we can distinguish two usages \textit{saving} and \textit{retrieving} files.

\subsection{File saving}
In order to add a file \textit{F} owned by client \textit{C} and process by a set of nodes \textit{N}(the system) :
\begin{enumerate}
 \item Optional : Check permission(addition one)
 \item Deduplication on C
 \item If no file duplication then :
 \begin{itemize}
  \item Data placement in N
  \item Data réplication in N and chunk deduplication
  \item Increase by 1 \textit{chunk ref} for each chunk
  \item Check data intégrity for each chunk
  \item Optional : Metadata management in N
  \item Metadata management in C
  \item Optional : Check metadata intégrity in C( optionaly on N)
 \end{itemize}
 \item Else:
 \begin{itemize}
  \item Increase by 1 \textit{chunk ref} for each chunk
  \item Metadata management in C
 \end{itemize}
\end{enumerate}

\textbf{Nota bene}
Addition perm != Access perm
With addition perm we can not access data, only add data or increase \textit{chunk ref}
It is the same with deletion perm.

\textbf{Nota Bene}
We can add a system for permissions( only usefull if the data on nodes are secured, only if the nodes can be trusted). If so, 
the permissions can be encode in objects Id for instance. So there is no need to check permissions : we encrypt some thing with 
a rsa key in the ids.

\subsection{File retrieving}%faute
In order to retrieve a file \textit{F} of Id \textit{I} on a client \textit{C}, stored on a set of 
nodes \textit{N} : 
\begin{enumerate}
 \item Optional : Check permission(acces one) for each chunk
 \item Optional : get metadata from C using \textit{I}
 \item Request data from N using metadata and \textit{I}
 \item Check data integrity
 \item Rebuild file
 \item Check file integrity
\end{enumerate}

\subsection{File deletion}
In order to delete a file \textit{F} of Id \textit{I} on a client \textit{C}, 
stored on a set of 
nodes \textit{N} :
\begin{enumerate}
 \item Optional : check permissions(deletion one)
 \item Optional : get metadata from C using \textit{I}
 \item Check metadata integrity
 \item Deletion in N( ie decrease \textit{chunk ref} and if 
  \textit{chunk ref} == 0 then remove chunk)
\end{enumerate}

\subsection{Get hard metadata}
??

\subsection{Get soft/abstract metadata}
??

\section{Deduplication}
We will discuss about the deduplication system, it is a client side service.

\subsection{}
We need id def and other thing do after
\section{The Object Naming Protocol(Convention ?)}

\textbf{se renseigner sur les différents systèmes de gestions de droits}

In this section we will discuss how to name( assign an identifier) to an object, can be a chunk or a file.

Let D : the data of the object
The Uniq Identifier Label :\\

schéma nécessaire

\begin{tabular}{c|c|p{0.7\textwidth}|}
4 Bytes & Command & 
\begin{enumerate}
 \item 1 byte : number of metadata indirection( nom un peu moche)
 \begin{itemize}
  \item For a default chunk : 0
  \item Otherwith see(le truc qui parle des métadonnées)
 \end{itemize}

\end{enumerate}\\
32 Bytes & Data hash & SHA256(D)\\

32/64 Bytes & Permission & ???????????????????????????????The encrypt version of the \textit{Data hash} je pense que c'est une très mauvaise idée à voir\\
\end{tabular}
\textbf{Remarque :}
Lets discuss about check sum, we will use CRC32(not 64) because it will be use on chunks, 
with a size less than ?(8KB).
Size overhead 10\%, is that worth it??????????????????? use to check hardware error, network 
error and not malicious action. But faster than sha256.\\\\
\textbf{Nota bene :}
We ought to ensure that $p(n,d) \leq 10^{-18}$, hence we can use SHA256 because storage system 
upper bound is about $2^{18}$Bytes.

\begin{proof}
The birthday problem can be generalized as follows given n random integers drawn from a 
discrete uniform distribution with range [1,d], what is the probability p(n;d) that at 
least two numbers are the same? (d=365 gives the usual birthday problem.)\\
   $p(n;d)\approx 1-e^{-n(n-1)/(2d)}$\\
   $p(n;d)\approx 1-\left({\frac {d-1}{d}}\right)^{n(n-1)/2}$\\\\
   \begin{table}[h]
      \centering
      \begin{tabular}{|c|c|c|c|c|c|c|c|c|}
	\hline
	Bits & Possible outputs &  \multicolumn{7}{c|}{Desired probability of random collision} \\
	\hline
	    & & $10^{-18}$ & $10^{-15}$ & $10^{-12}$ & $10^{-9}$ & $10^{-6}$ & 0.1\% & 1\% \\
	\hline
	16 & 65536  & $\le2$& $\le2$& $\le2$& $\le2$& $\le2$& 11 & 36  \\
	32 & $4.3*10^{9}$ & $\le2$& $\le2$& $\le2$& 3 & 93 & 2900 & 9300 \\
	64 & $1.8*10^{19}$ & 6 & 190 & 6100 & 190000 & 6100000 & $1.9*10^{8}$ & $6.1*10^{8}$ \\
	128 & $3.4*10^{38}$ & $2.6*10^{10}$ & $8.2*10^{11}$ & $2.6*10^{13}$ & $8.2*10^{14}$ & $2.6*10^{16}$ & $8.3*10^{17}$ & $2.6*10^{18}$\\
	256 & $1.2*10^{77}$ & $4.8*10^{29}$ & $1.5*10^{31}$ & $4.8*10^{32}$ & $1.5*10^{34}$ & $4.8*10^{35}$ & $1.5*10^{37}$ & $4.8*10^{37}$\\
	384 & $3.9*10^{115}$ & $8.9*10^{48}$ & $2.8*10^{50}$ & $8.9*10^{51}$ & $2.8*10^{53}$ & $8.9*10^{54}$ & $2.8*10^{56}$ & $8.9*10^{56}$\\
	512 & $1.3*10^{154}$ & $1.6*10^{68}$ & $5.2*10^{69}$ & $1.6*10^{71}$ & $5.2*10^{72}$ & $1.6*10^{74}$ & $5.2*10^{75}$ & $1.6*10^{76}$\\
	\hline
    \end{tabular}
    \caption{}
   \end{table}
   
ref:    https://fr.wikipedia.org/wiki/Paradoxe\_des\_anniversaires
  
\end{proof}


\textbf{Problem}
The permission handler as only one mod ie acces(write/read) or nothing. Other with data permission must be ensure by the 
metdata owner.
\section{The Node Labeling System}
So a node have a uniq label composed of 41 Bytes.

schéma nécessaire
\begin{table}[h]
  \centering
  \begin{tabular}{|c|c|p{0.3\textwidth}|p{0.4\textwidth}|}
  \hline
  Size & Name & Descritpion & Infos \\
  \hline
  8 Bytes & Number of nodes & This is the current number of nodes & use for system reorganisation see ?\\
  32 Bytes & Inner Id & SHA226(Ipv6+Port number) & 
    Ipv6+Port number : uniq identifier,
    we hash it to have a uniform distribution in order to avoid same ipv6 and 
    different port to be "near"(for fault tolerance purposes)
    if nap we must add some randomness in id(and we must check system build on deterministic id).......\\
  1 Bytes & Re-labeling flag & Indicate if a node was relabeled & True relabeled/ false real id\\
  \hline
  \end{tabular}
  \caption{}
\end{table}


\textbf{il faudra penser à un system optionnel d'authentification des noeuds, 
et des communications}
\section{Metadata management/handling}
\subsection{Hard metadata}
Optional
"Hard metadata" are the metadata needed to rebuild a file from a File ID(it is a Object ID, indeed SHA256(D) ensure unicity)\\

After déduplication we obtains a list of chunks, a 4-uplets list (\textit{pos, length, id, data}).\\

\begin{table}[h]
  \centering
  \begin{tabular}{|c|c|c|}
    \hline
    \textit{id} & ? see object naming & chunk's id, using Object Naming Protocol\ref{}\\
    \textit{pos} & 8 Bytes & Begi"nn"ing of the chunk's data in the original file\\
    \textit{length} & 8 Bytes & Length( bytes) of the chunk's data\\
    \textit{checksum} & 4 Bytes & CRC32-C\\
    \textit{data} & - & chunks' data\\
    \hline
  \end{tabular}
  \caption{}
\end{table}

schema d'explication des niveaux d'indirection
et rapide calcul de la taille max des fichiers stockés

The improvement of the system of hard metadata : it is fully "transparant en terme d'architectur même service que la sauvegarde des chunks".

\textbf{Nota Bene :} To rebuild a file we have only have to use the file retrieving API( chunks number)


\subsection{Soft/abstract metadata}

\section{Hyper-cube properties}
\begin{definition}
  Let $\forall i \in [|0,n-1|]$, $e_{i} = (0,...,0,1,0,...,0)$ ie $\bar{e_{i}} = 2^{i}$ \\
  $$Q_{n} = \left\{ \sum_{i=0}^{n-1}{ \delta_{i}e_{i} } = (\delta_{0}, ..., \delta_{n-1}) | (\delta_{0}, ..., \delta_{n-1}) \in \{0,1\}^{n} \right\}$$
\end{definition}
\begin{remarque} 
  Let $n \in \mathbb{N}^{*}, \forall u \in Q_{n}, \bar{u} \in [|0, 2^{n}-1|]$ and $\bar{u} = \sum_{i=0}^{n-1}{ \delta_{i}\bar{e_{i}} }$
\end{remarque}

\begin{definition}%//https://en.wikipedia.org/wiki/Hypercube\_graph
  The hypercube graph Qn may be constructed from the family of subsets of a set with n elements, by making a vertex for each 
  possible subset and joining two vertices by an edge whenever the corresponding subsets differ in a single element. 
  Equivalently, it may be constructed using 2n vertices labeled with n-bit binary numbers and connecting two vertices by an 
  edge whenever the Hamming distance of their labels is one. These two constructions are closely related: a binary number may 
  be interpreted as a set (the set of positions where it has a 1 digit), and two such sets differ in a single element whenever 
  the corresponding two binary numbers have Hamming distance one.\\\\
  Let $n \in \mathbb{N}^{*}$ and $H_{n}=(V_{n}, E_{n}, Id_{n}) = (Q_{n}, Id_{n})$ an hyper-cube of dimension n\\
  Such that :
  \begin{tabular}{cc}
    $V_{n}$ & set of vertices\\
    $E_{n}$ & set of edges\\
    $Id_{n}$ & $V \rightarrow \bar{Q_{n}}=[|0, 2^{n}-1|]$, bijective such that $Id_{n}(V_{n})=Q_{n}$\\
    $H_{n}$  & $\left\{ (u,v) 
		\begin{array}{l}
		  (u,v) \in Q_{n}^{2}\\
		  d_{ham}(u,v) = 1\\
		\end{array}
		\right\}$\\
  \end{tabular}
\end{definition}


\begin{minipage}{0.49\textwidth}
 \begin{center}
 \textbf{Counting properties :}\\
 \begin{tabular}{|c|c|}
  \hline
    Number of vertices & $2^{n}$\\
    Number of edges & $n2^{n-1}$\\
    Diameter & $n$\\
    Girth & $4\ if\ n \geq 2$\\
    Chromatic number & 2\\
    Spectrum 	& ${\displaystyle \{(n-2k)^{\binom {n}{k}};k=0,\ldots ,n\}}$\\
  \hline
  \end{tabular}
  \end{center}
\end{minipage}
\begin{minipage}{0.49\textwidth}
 \begin{center}
  \textbf{Properties :}\\
 \begin{tabular}{|c|}
  \hline
  Symmetric\\
  Distance regular\\
  Unit distance\\
  Hamiltonian\\
  Bipartite\\
  \hline
  \end{tabular}
  \end{center}
\end{minipage}

\begin{lemma}
 Let $n \in \mathbb{N}^{*}$, $|faces(Q_{n})| = 2^{(n-2)}\binom{n}{2}$
\end{lemma}

% ici un algo qui calcules les faces(deterministe et linaire en 2^{n}), faudrait remplacer le n par un $\alpha$ dans cette section

\begin{algorithmic}
    \State Open = $Q_{n}$
    \State Closed =  $\varnothing$
    \State Faces = $\varnothing$
    \While{ $Open \neq \varnothing$ }
      \State $i_{1}, ..., i_{n} = n-min( Open )$
      \For{ $0 < j \leq n$}
	\State $i_{1}, i_{j-1}, k, i_{j+1} i_{n} = n-min( Open )$
      \EndFor
      \State $ $
    \EndWhile
    \If {$s = 0$}

    \State $i\gets random(0, deg(u)-1)$
    \State $s \gets deg(u)$
    \State $move\_to( end(v,i) )$
       \Else
    \State $p \gets \min_{ \frac{s}{deg(u)}, 1}$
    \If {$random() \leq p$}
      \State $s \gets 0$
    \Else
      \State $reverse\_move$
    \EndIf
      \EndIf
 \end{algorithmic}

\begin{lemma}
  Let $n \in \mathbb{N}^{*}$, $(u,v) \in Q_{n}^{2}$ then (u,v) are n-connected.
\end{lemma}

\begin{proof}%http://math.stackexchange.com/questions/305088/proof-that-a-n-hypercube-is-n-vertex-connected#305136
  Let $u=(u_{0}, u_{1}, u_{n-1})$ and $v=(v_{0}, v_{1}, v_{n-1})$ \\
  let us prove that by induction on $n \in \mathbb{N}^{*}$\\
  \begin{itemize}
   \item Case n=1 it works
   \item Case n+1 assuming that it works for n
   \begin{enumerate}
    \item Case 1 : $\exists i \in [|0,n-1|]$, $u_{i}=v_{i}$ \\
    Let $u_{|i}=(u_{0}, ..., u_{i-1}, u_{i+1}, ..., u_{n-1})$(resp $v_{|i}$) then by definition $(u_{|i}, v_{|i}) \in Q_{n-1}$\\
    Hence by induction property (u,v) are (n-1)-connected in $Q_{n-1}$ ; therefor there is (n-1) vertice disjoint paths between u and v, noted $P_{n-1}^{|i}(u,v)$\\
    Now we extend the paths as follow : 
    $$
    P_{n}(u,v) = \left\{p = (s^{1}, ..., s^{m}),
    \begin{array}{lcl}
        p_{|i}=(s_{|i}^{1}, ..., s_{|i}^{m}) \in P_{n-1}^{|i}(u,v) \cap Q_{n-1}^{m} & with & s_{|i}^{j}= (s_{0}, ..., s_{i-1}, s_{i+1}, ..., s_{m}\\
        s^{j}= (s_{0}, ..., s_{i-1}, u_{i}, s_{i+1}, ..., s_{m}) & with & u_{i}=v_{i}\\
    \end{array}
    \right\}
    $$
    
    $P_{n-1}^{|i}(u,v)$ is a set vertice disjoint paths so $P_{n}(u,v)$ too
   \item Case 2: $\forall i \in [|0,n-1|]$, $u_{i} \neq v_{i}$ \\
   Let $e_{j} = (0,...,0,1,0,...,0)$ ie $\bar{e_{j}} = 2^{j}$ and $u \leftrightarrow e_{j}= \bar{u} \text{ xor } \bar{e_{j}}$ \\
   $$
   P_{n}(u,v) = \left\{
   (s_{0}, s_{1}, ..., s_{n-1}),
   \begin{array}{c}
   s_{0}=(u \leftrightarrow e_{j})\\
   s_{1}=(s_{0} \leftrightarrow e_{j+1})\\
   ...\\
   s_{n-j}=(s_{n-j-1} \leftrightarrow e_{n})\\
   s_{n-j+1} =(s_{n-j}\leftrightarrow e_{0})\\
   ...\\
   s_{n-1} = (s_{n-2}\leftrightarrow e_{j-1})\\
   \end{array}
   \right\}
   $$
   \end{enumerate}
  \end{itemize}
\end{proof}

\begin{lemma}
  $N_{d=i}(u) =$ neighbours of u at distance i\\
  $i \le 4$ and $j \le 4$\\
  $$
    P_{n}^{d=(i,j)}(u,v) = \bigsqcup_{
    \begin{array}{l}
     u' \in N_{d=i}(u) \\
     v' \in  N_{d=j}(v)\\
    \end{array}
    }{P_{n}(u',v')}
  $$
\end{lemma}

\begin{lemma}
Let $n \in \mathbb{N}, i\in [|1,n-1|]$, to disconnect a $H_{i}$ in $H_{n}$ we have to kill $(n-i)2^{i}$ nodes.\\
\end{lemma}

\begin{proof}
Without loss of generality, we can express $Q_{i}$ as $Q_{n}$ subset with some prefix of length $n-i$ \\
Let us choose one prefix : $pref=\sum_{j=i+1}^{n-1}{ \delta_{j}e_{j} }$ with $(\delta_{i+1}, ..., \delta_{n-1}) \in \{0,1\}^{n-i}$ \\
Let
$\begin{array}{ccccc}
f & : & Q_{i} & \to & Q_{n} \\
 & & s=\sum_{j=0}^{i}{ \delta_{j}e_{j} } & \mapsto & s'= pref + s \\
\end{array}$
$Q'_{i} = f(Q_{i})$ and $|Q'_{i}| = |Q_{i}|$\\\\
  To disconnect $Q'_{i}$ from others we have to kill all nodes in $N( Q'_{i}) \backslash\lbrace{Q'_{i}}\rbrace$\\

  \begin{tabular}{lcl}
   $N( Q'_{i}) \backslash\lbrace{Q'_{i}}\rbrace$ & $=_{def}$ & $\{ u | d_{ham}(u,Q'_{i})=1\}$\\
	& $=$ & $\{ u | u = pref' + v, v \in Q_{i}, d_{ham}(pref,pref')=1 \}$\\
  $|N( Q'_{i}) \backslash\lbrace{Q'_{i}}\rbrace|$ & $=$ & $|\{pref'| d_{ham}(pref,pref')=1 \}| + |Q_{i}|$\\
	& $=$ & $(n-i) 2^{i}$\\
  \end{tabular}
\end{proof}

\begin{definition} Let $n \in \mathbb{N}^{*}$\\
  Let $u \in Q_{n}$, $F_{u}$ is the state variable of u
  $$F_{u} : \left\{ 
  \begin{array}{cc}
   F_{u}=0 & \text{u has failed  at time 0}\\
   F_{u}=1 & \text{u has failed  at time 1}\\
   ... & ...\\
   F_{u}=i & \text{u has failed  at time i}\\
  \end{array}
  \right\}
 $$\\
 Let $\lambda \in ]0,1[$, $\forall t \in \mathbb{N}, \forall u \in Q_{n}, (F_{u}=t) = \lambda$ \\
 More over we assume that $\forall (u,v) \in Q_{n}$, $u \neq v$ then $F_{u}, F_{v}$ are pairwise independent\\
 
 Model durint transition t to t+1 there are not failure and all dead are replace
\end{definition}

\begin{remarque}
  It is an uniform law, and it's not accurate probability increase with time and there is renewal
\end{remarque}

To redo with better modelisation!!!!!!!!!!!!!!!!!!!!!!!!!!!!!!!!!!!!
\begin{prop} 
  Let $n \in \mathbb{N}^{*}, i \in [|1,n-1|], t \in [|1,n-1|]$, $p(Disconnect_{H(i)}=t)= \lambda ^{(n-i)2^{i}}$
\end{prop}

\begin{proof}Let $n \in \mathbb{N}^{*}, i \in [|1,n-1|], t \in [|1,n-1|]$\\
  \begin{tabular}{lll}
    $p(Disconnect_{H(i)})$ & $=$ & $p(F_{N( Q'_{i}) \backslash\lbrace{Q'_{i}}\rbrace}=t)$\\
    & $=_{independent}$ & $\Pi_{u \in N( Q'_{i}) \backslash\lbrace{Q'_{i}}\rbrace}{p(F_{u}=t)}$\\
    & $=_{independent}$ & $\lambda^{ N( Q'_{i}) \backslash\lbrace{Q'_{i}}\rbrace }$\\
    & $=$ & $\lambda^{ (n-i) 2^{i} }$\\
     \end{tabular}
\end{proof}


\section{The Hypercube Placement Protocol}
\subsection{Services}
\begin{itemize}
 \item Store objects
 \item Restore objects
 \item Remove objects
 \item Optionnaly handle metadata
\end{itemize}


\subsection{Specifications}
\begin{minipage}{0.49\textwidth}
    \begin{tabular}{|c|}
      \hline
      Critical\\
      Production\\
      Feature\\ 
      \hline
    \end{tabular}
\end{minipage}
\begin{minipage}{0.49\textwidth}
    \begin{tabular}{|c|c|}
      \hline
      0 & Nothing done\\
      1 & Specification\\
      2 & Algorithm\\
      3 & Implementation\\ 
      4 & Production\\
      \hline
     \end{tabular}
\end{minipage}




\begin{table}[h]
  \centering
  \begin{tabular}{|c|c|c|c|}
    \hline
    Service name & Description & Utility & State \\
    \hline
    Object placement & & Critical & 0 \\
    Object access & & Critical & 0 \\
    Object deletion & & Production & 0\\
    Data replication & & Critical & 0 \\
    Data integrity check & & Critical & 0 \\ %on each transaction
    Node integrity check & & ? & 0 \\
    Failure detection & & Critical & 0 \\
    Node integrity update & & ? & 0 \\
    Load balancing & & Critical & 0 \\
    Data balancing & & Critical & 0 \\
    Node addition & & Critical & 0 \\
    Node deletion & & Critical & 0 \\
    Quota/Permission & & Production & 0 \\
    File versionning & Here or in metadata handling(delta chunking) ??? & ? & ? \\
    \hline
  \end{tabular}
  \caption{}
\end{table}

\subsection{Local Complexity requirements}
In a \textbf{n} nodes system, \textbf{m} number of objects:\\

\begin{table}[h]
  \centering
  \begin{tabular}{|c|c|c|}
  \hline
  Description & Time & Space\\
  \hline
  Object placement & $O(\ln(n))$ &  $O(\ln(n))$ \\
  Object access & $O(\ln(n))$ &  $O(\ln(n))$ \\
  Object deletion & $O(\ln(n))$ & $O(\ln(n))$\\
  Data replication & O(max( object placement, object access)) & O(max( object placement, object access))\\
  Data integrity check & O(1) & O(1) \\
  Load balancing & $O(\ln(n))$ &  $O(\ln(n))$ \\
  Data balancing & $\frac{1}{m}$ migrations & $\frac{1}{m}$ migrations\\ %when a node is over filled
  Node addition & $\frac{1}{m}$ migrations & $\frac{1}{m}$ migrations\\
  Node deletion & $\frac{1}{m}$ migrations & $\frac{1}{m}$ migrations\\
  Node integrity check & ? & ? \\ % we assume node still alive, if not it's node deletion
  Node map(store on a node) & ? & $O(1)$ \\
  Node map update & $O(1)$ & $O(1)$ \\
  Node state & ? & $O(\ln(n))$ \\
  Node state update & $O(\ln(n))$ & ? \\
  \hline
  \end{tabular}
  \caption{}
\end{table}
Doit on décrire les différents sytème de cache ?? Filtre de bloom, cache in memmory 1/3 etc ??

\subsection{Global Complexity}
\begin{table}[h]
  \centering
  \begin{tabular}{|c|c|c|}
  \hline
  Description & Time & Space\\
  \hline
  Object placement & $O(\ln(n))$ messages propagation &  $O(n)$ messages  \\
  Object access & $O(\ln(n))$ messages propagation &  $O(n)$ messages \\
  Object deletion & $O(\ln(n))$ messages propagation &  $O(n)$ messages \\
  Data replication & O(max( object placement, object access)) & O(max( object placement, object access))\\
  Data integrity check & O(1) & O(1) \\
  Load balancing & $O(\ln(n))$ messages propagation &  $O(n)$ messages \\
  Data balancing & $\frac{1}{m}$ migrations & $\frac{1}{m}$ migrations\\ %when a node is over filled
  Node addition & $\frac{1}{m}$ migrations & $\frac{1}{m}$ migrations\\
  Node deletion & $\frac{1}{m}$ migrations & $\frac{1}{m}$ migrations\\
  Node map update & $O(\ln{n})$ messages propagation & $O(n)$ messages\\
  Node state update & $O(\ln{n})$ messages propagation & $O(n)$ messages\\
  \hline
  \end{tabular}
  \caption{}
\end{table}


\section{The Global Node Map}
\

\subsection{Nodes clustering}
  \begin{figure}[h]
    \label{node-clustering}
    \centering
    \scalebox{0.9}{
      \begin{tikzpicture}

\draw (0,0) -- (16,0) -- (16, -0.3) -- (0, -0.3) -- (0,0);
\node[draw=none, red] at (8, -0.2) {domain of nodes' ID number/label};
\node[draw=none] at (-1, -0.15) {lvl 0};

\foreach \i in {0.1,0.2,...,15.9} {
    \draw (\i,0) -- (\i, -0.3); 
}

\node[draw=none] at (-1, -0.5) {lvl 1};
\foreach \i in {0,2,...,14} {
  \draw (\i + 0.05, -0.4) -- (\i + 0.05, -0.5) -- (\i + 1.95, -0.5) -- (\i + 1.95, -0.4);
  \node[draw=none] at (\i + 1, -0.7) {$k = 2^{\alpha}$};
}

\node[draw=none] at (-1, -0.9) {lvl 2};
\foreach \i in {0,4,8,12} {
  \draw (\i + 0.025, -0.8) -- (\i + 0.025, -0.9) -- (\i + 3.95, -0.9) -- (\i + 3.95, -0.8);
  \node[draw=none] at (\i + 2, -1.1) {$f=2^{\beta}$};
}

\foreach \i in {0,0.2,0.4,...,16} {
  \draw (\i,-1.3) -- (\i+0.05, -1.3) ;
  \draw (\i,-1.4) -- (\i+0.05, -1.4) ;
  \draw (\i,-1.5) -- (\i+0.05, -1.5) ;
}

\node[draw=none] at (-1, -1.7) {lvl $l_{max}-1$};
\foreach \i in {0,8} {
  \draw (\i + 0.025, -1.6) -- (\i + 0.025, -1.7) -- (\i + 7.95, -1.7) -- (\i + 7.95, -1.6);
  \node[draw=none] at (\i + 3, -1.9) {$f=2^{\beta}$};
}

\node[draw=none] at (-1, -2.1) {lvl $l_{max}$};
\draw (0.025, -2) -- (0.025, -2.1) -- (15.95, -2.1) -- (15.95, -2);
\node[draw=none] at (8, -2.3) {$f=2^{\beta}$};

\end{tikzpicture}
    }
    \caption{2-2-hyper cube structure}
  \end{figure}
  
The main idea is to split nodes in \textbf{k}-nodes cluster and to aggregate such cluster in second order cluster and so on( 
in a fractal mod). In this papers we will consider a cluster as an Hyper Cube where vertex are node and edges is connection 
between node( UDP/TCP connections). \\

We split the node-ids domain in buckets of \textbf{k} nodes( that make a cluster) then we split we buckets domain in hight order buckets
of \textbf{f} low order buckets and we repeat. 

\subsection{Global Map}
\begin{minipage}{0.49\textwidth}
   \centering
   \scalebox{0.3}{
       \begin{tikzpicture}
   \node[point] (a) {};
   \node[point, below right=of a] (a1) {};
   \node[node, below right=of a1] (0) {};
   \node[node, right=of 0](1) {};
   \node[node, below=of 1](2) {};
   \node[node, left=of 2] (3) {};
   \draw (3) -- (2) -- (1) -- (0) -- (3);
   
   \node[node, right=of 1] (4) {};
   \node[node, right=of 4] (5) {};
   \node[node, below=of 5] (6) {};
   \node[node, left=of 6] (7) {};
   \draw (7) -- (6) -- (5) -- (4) -- (7);
   \draw[dashed] (1) -- (4);
   \draw[dashed] (2) -- (7);
   
   \node[point, above right=of 5] (b1) {};
   \node[point, right=of b1] (a2) {};
   \node[node, below right=of a2] (8) {};
   \node[node, right=of 8] (9) {};
   \node[node, below=of 9] (10) {};
   \node[node, left=of 10] (11) {};
   \draw (11) -- (10) -- (9) -- (8) -- (11);
   \draw[dashed] (5) -- (8);
   \draw[dashed] (6) -- (11);
   
   \node[node, right=of 9] (12) {};
   \node[node, right=of 12] (13) {};
   \node[node, below=of 13] (14) {};
   \node[node, left=of 14] (15) {};
   \draw (15) -- (14) -- (13) -- (12) -- (15);
   \draw[dashed] (9) -- (12);
   \draw[dashed] (10) -- (15);
   \node[point, above right=of 13] (b2) {};
   \node[point, above right=of b2] (b) {};
   
   %% Deuxième ligne
   \node[node, below=of 3] (16) {};
   \node[node, right=of 16](17) {};
   \node[node, below=of 17](18) {};
   \node[node, left=of 18] (19) {};
   \draw (19) -- (18) -- (17) -- (16) -- (19);
   \node[point, below left=of 19] (d1) {};
   \node[point, below=of d1] (a3) {};
   
   \node[node, right=of 17] (20) {};
   \node[node, right=of 20] (21) {};
   \node[node, below=of 21] (22) {};
   \node[node, left=of 22] (23) {};
   \draw (23) -- (22) -- (21) -- (20) -- (23);
   \draw[dashed] (17) -- (20);
   \draw[dashed] (18) -- (23);
   \draw[dashed] (3) -- (16);
   \draw[dashed] (2) -- (17);
   \draw[dashed] (7) -- (20);
   \draw[dashed] (6) -- (21);
   
   

   \node[node, below=of 11] (24) {};
   \node[node, right=of 24] (25) {};
   \node[node, below=of 25] (26) {};
   \node[node, left=of 26] (27) {};
   \draw (27) -- (26) -- (25) -- (24) -- (27);
   \draw[dashed] (21) -- (24);
   \draw[dashed] (22) -- (27);
   \draw[dashed] (11) -- (24);
   \draw[dashed] (10) -- (25);
   
   \node[node, right=of 25] (28) {};
   \node[node, right=of 28] (29) {};
   \node[node, below=of 29] (30) {};
   \node[node, left=of 30] (31) {};
   \draw (31) -- (30) -- (29) -- (28) -- (31);
   \draw[dashed] (25) -- (28);
   \draw[dashed] (26) -- (31);
   \draw[dashed] (14) -- (29);
   \draw[dashed] (15) -- (28);
   
   \node[point, below right=of 30] (c2) {};
   \node[point, below=of c2] (b4) {};
   \node[point, below right=of 22] (c1) {};
   \node[point, right=of c1] (d2) {};
   
   %%Trosième ligne
   \node[node, below right=of a3] (32) {};
   \node[node, right=of 32](33) {};
   \node[node, below=of 33](34) {};
   \node[node, left=of 34] (35) {};
   \draw (35) -- (34) -- (33) -- (32) -- (35);
   \draw[dashed] (18) -- (33);
   \draw[dashed] (19) -- (32);

   
   \node[node, right=of 33] (36) {};
   \node[node, right=of 36] (37) {};
   \node[node, below=of 37] (38) {};
   \node[node, left=of 38] (39) {};
   \draw (39) -- (38) -- (37) -- (36) -- (39);
   \draw[dashed] (33) -- (36);
   \draw[dashed] (34) -- (39);
   \draw[dashed] (22) -- (37);
   \draw[dashed] (23) -- (36);
   \node[point, above right=of 37] (b3) {};
   \node[point, right=of b3] (a4) {};
   
   

   \node[node, below right=of a4] (40) {};
   \node[node, right=of 40] (41) {};
   \node[node, below=of 41] (42) {};
   \node[node, left=of 42] (43) {};
   \draw (43) -- (42) -- (41) -- (40) -- (43);
   \draw[dashed] (37) -- (40);
   \draw[dashed] (38) -- (43);
   \draw[dashed] (27) -- (40);
   \draw[dashed] (26) -- (41);
   
   \node[node, right=of 41] (44) {};
   \node[node, right=of 44] (45) {};
   \node[node, below=of 45] (46) {};
   \node[node, left=of 46] (47) {};
   \draw (47) -- (46) -- (45) -- (44) -- (47);
   \draw[dashed] (41) -- (44);
   \draw[dashed] (42) -- (47);
   \draw[dashed] (31) -- (44);
   \draw[dashed] (30) -- (45);
   
   %%Quatrième ligne
   \node[node, below=of 35] (48) {};
   \node[node, right=of 48] (49) {};
   \node[node, below=of 49] (50) {};
   \node[node, left=of 50] (51) {};
   \draw (51) -- (50) -- (49) -- (48) -- (51);
   \draw[dashed] (34) -- (49);
   \draw[dashed] (35) -- (48);
   \node[point, below left=of 51] (d3) {};
   \node[point, below left=of d3] (d) {};
   
   \node[node, right=of 49] (52) {};
   \node[node, right=of 52] (53) {};
   \node[node, below=of 53] (54) {};
   \node[node, left=of 54] (55) {};
   \draw (55) -- (54) -- (53) -- (52) -- (55);
   \draw[dashed] (49) -- (52);
   \draw[dashed] (50) -- (55);
   \draw[dashed] (38) -- (53);
   \draw[dashed] (39) -- (52);
   \node[point, below right=of 54] (c3) {};
   \node[point, right=of c3] (d4) {};
   
   
   \node[node, below=of 43] (56) {};
   \node[node, right=of 56] (57) {};
   \node[node, below=of 57] (58) {};
   \node[node, left=of 58] (59) {};
   \draw (59) -- (58) -- (57) -- (56) -- (59);
   \draw[dashed] (43) -- (56);
   \draw[dashed] (42) -- (57);
   \draw[dashed] (53) -- (56);
   \draw[dashed] (54) -- (59);
   
   \node[node, right=of 57] (60) {};
   \node[node, right=of 60] (61) {};
   \node[node, below=of 61] (62) {};
   \node[node, left=of 62] (63) {};
   \draw (63) -- (62) -- (61) -- (60) -- (63);
   \draw[dashed] (57) -- (60);
   \draw[dashed] (58) -- (63);
   \draw[dashed] (47) -- (60);
   \draw[dashed] (46) -- (61);
   
   \node[point, below right=of 62] (c4) {};
   \node[point, below right=of c4] (c) {};
   
   %second order hcube
   \draw[green] (a1) -- (b1) -- (c1) -- (d1) -- (a1);
   \draw[green] (a2) -- (b2) -- (c2) -- (d2) -- (a2);
   \draw[green] (a3) -- (b3) -- (c3) -- (d3) -- (a3);
   \draw[green] (a4) -- (b4) -- (c4) -- (d4) -- (a4);
   \draw[red] (a) -- (b) -- (c) -- (d) -- (a);
  \end{tikzpicture}
  
   }
   \captionof{figure}{2-2-hyper cube structure}
\end{minipage}
\begin{minipage}{0.49\textwidth}
So we can see the network, at any level, as an Hyper Cube where vertices are hypercube(recursively).
\end{minipage}\\\\
\begin{table}[h]
  \centering
  \begin{tabular}{|c|c|c|c|c|}
    \hline
    layer & Description & Number of vertices &  Number of nodes & Dimension\\
    \hline
    0		   	& node & 1 & 1 & 0\\
    1		   	& $\alpha-dim$ hcube(vertices are nodes) & f & $f=2̂^{\alpha}$ & $\alpha$\\
    l		   	& $\alpha-dim$ hcube(vertices are $\alpha-dim$ hcube) & $f$ & $f^{l} = 2^{\alpha l}$  & $\alpha$\\
    \hline
  \end{tabular}
  \caption{}
\end{table}

\begin{definition} Global map\\
  $$\text{Let }G^{f}(n), (f,n) \in \mathbb{N} \times \mathbb{N}^{*}, \text{ a global map where }
  \left\{\begin{array}{ll}
     f = 2^{\alpha} & \text{size of a cluster}\\
     n 	& \text{number of nodes}\\ 
     l(n) &  \text{number of layers of hypercube}\\
    \alpha & \text{ mesh parameter}\\
  \end{array}\right.$$\\
  \begin{tabular}{cc}
    let $G^{f}(n)=_{def} (V^{f}(n),E^{f}(n), Id_{f}) =_{def} (Q^{f}(n), Id^{f})$
  \end{tabular}
\end{definition}



\begin{definition} Let $(f,l) \in \mathbb{N} \times \mathbb{N}^{*}$, $G_{l}^{f} = \left\{ G^{f}(n) |n \in \mathbb{N}, l(n) = l \right\}$\\
\end{definition}

\begin{definition} Let $n \in \mathbb{N}^{*}$, $l \in [|1, l(n)|]$, 
$H_{l+1}^{f}(n) = (V_{l+1}^{f}(n), E_{l+1}^{f}(n), Id_{l+1}^{f})$\\
Where $V_{l+1}^{f}(n) = \left\{V_{0}^{f}(n), ..., V_{f-1}^{f}(n)) \right\} \subset (Q_{l}^{f})^{f} $ and $Id_{l+1}^{f}(V_{i}^{f}(n)	)=i$\\
$\forall v \in V, V_{l}^{f}(v) = V_{i}^{f}(n)$ where $i \in [|0,f-1|]$ such that $v \in V_{i}^{f}(n)$, i is uniq
\end{definition}

\begin{definition} global identifier\\
  Let $(f,n) \in \mathbb{N} \times \mathbb{N}^{*}$, and $v \in V$:\\
      $gid_{G^{f}(n)}(v) = gid_{l(n)}(v) \text{ where } \left\{ 
	\begin{array}{llll}
	 gid_{1}(v) & = & Id_{f}(v) & \\
	 gid_{l+1}(v) & = & Id_{l+1}^{f}(V_{l}^{f}(v)) \times 2^{(l-1)\alpha+\alpha} + gid_{l}(v) & | 0 \leq l \le l(n)\\
	\end{array}\right\}$\\  
\end{definition}

\begin{prop}
Let $(f,n) \in \mathbb{N} \times \mathbb{N}^{*} gid_{G^{f}(n)}(v) =\sum_{l=1}^{l(n)-1} Id_{l}^{f}(V_{l}^{f}(v)) \times 2^{(l-1)\alpha}$
\end{prop}


\begin{definition}[ neighbourhood] Let $(f,n) \in \mathbb{N} \times \mathbb{N}^{*}, i \in Q^{f}(n)$,\\
$N^{f}(i) =_{def}\bigcup_{l=1}^{l(n)}{N_{l}^{f}(v)}$\\
$where \left\{
\begin{array}{lll}
 N_{1}^{f} & = &\{j|j\in Q^{f}(n), d_{ham}(i,j)=1\}\\
 N_{1}^{f} & = &\{j|j\in Q^{f}(n), d_{ham}(i,j)=1\}\\
\end{array}\right\}$

\end{definition}


\begin{prop}
  Let $(f,n) \in \mathbb{N} \times \mathbb{N}^{*}$ then $l(n)-1 < \ln(\frac{|G^{f}(n)|}{f}-f) =_{def} \ln(\frac{n}{f}-f) \leq l(n)$\\ 
\end{prop}

\begin{proof}Let $(f,n) \in \mathbb{N} \times \mathbb{N}^{*}$\\
  \begin{itemize}
   \item Case 1 :   $Layers(G^{f}(n))$ is a perfect tree, hence there is $f^{l(n)}$ leaves(1-cluster)\\
      Moreover there is $\frac{|G^{f}(n)|}{f}=\frac{n}{f}$ 1-cluster(ie leaves by construct)\\
    \begin{tabular}{rcl}
      $f^{l(n)}$ & = & $\frac{n}{f}$ \\
      $l(n)\ln(f)$ & = & $\ln(\frac{n}{f})$\\
      $l(n)$ & = & $\ln(\frac{n}{f}-f)$\\ 
    \end{tabular}
   \item Case 2 :   $G^{f}(n)$ is a complete tree\\
     Therefore 
    \begin{tabular}{lclcl}
      $f^{l(n)-1}$ & $<$ & $\frac{|G^{f}(n)|}{f}$ & $\leq$ & $f^{l(n)}$\\
      $(l(n)-1)\ln(f)$ & $<$ & $\ln(\frac{n}{f})$ & $\leq$ & $l(n)\ln(f)$\\
      $l(n)-1$ & $<$ & $\ln(\frac{n}{f}-f)$ & $\leq$ & $l(n)$\\ 
    \end{tabular}
  \end{itemize}
\end{proof}


 
 
\begin{lemma} \label{lemma-min_connexity}
  Let $(k,f,d) \in \mathbb{N} \times \mathbb{N} \times \mathbb{N}^{*}$, $G^{f}$ is $\min(\ln(k),\ln(f))$-connected
\end{lemma}
 
 \begin{proof}
  
 \end{proof}
 
 \begin{lemma}
  Let blablabla, $\forall l \in [|2,lmax-1|]$, the network composed of l-hcube(ie l-hcube are vertices) is f-connexe.
 \end{lemma}

 \begin{proof}
 same proof ass \ref{lemma-min_connexity}
\end{proof}

\begin{lemma}
 If a node is in cluster at d distance of the fronter then it is $k*f^{d}$ connexe with >d distance of fronter.
 Note that d<=ln(n) (esperance en kf(n/2) ? )
\end{lemma}

\begin{proof}
 todo
\end{proof}

\begin{lemma}
 A node as exactly k+f neightbourgs if it is not in the fronteir and k< <k+f otherwith.
\end{lemma}

\begin{proof}
 Todo
\end{proof}


Lets describe the semantic of the network, vertices are nodes where data is/are? stored, edges is the allowed cmmunication 
channel used for all except data transmition. Indeed as soon as data location is computed both node open a direct channel, the network provide low overhead 
with node discovering, data placement, global node map modifications and in the other hand it provide very fast data transfert by bypassing the topologie of the network.
\\
We will call communication rounds : each $\tau$s (something like 0.1, a parameter) each node send data map modification to its neightbourg and heartbeat

\subsection{Object placement}
\begin{minipage}{0.49\textwidth}
   \centering
   \scalebox{0.1}{
     \begin{tikzpicture}
 
% 81 origin
% 27 lvl lmax
% 9 9 9 lvl i
% 3 3 3 lvl 1
% 111  lvl 0
\foreach \r in {3}{%ratio vertical
\foreach \i in {0,1,2} {
    \draw (27 * \i, \r * 0) -- (27 * \i, \r * -1) -- (27 * \i+27, \r * -1) -- (27 * \i+27, \r * 0) -- (27 * \i, \r * 0); 
    \draw (13.5 + 27 * \i, \r * -1) -- (13.5 + 27 * \i, \r * -2);
    \draw (4.5 + 27 * \i, \r * -2) -- (22.5 + 27 * \i, \r * -2);

        
      \foreach \j in {3*\i,3*\i+9,3*\i+18}{
	\draw (9 * \j + 4.5, \r * -2) -- (9 * \j + 4.5, \r * -3);
	\draw (9 * \j, \r * -4) -- (9 * \j, \r * -3) -- (9 * \j+9, \r * -3) -- (9 * \j+9, \r * -4) -- (9 * \j, \r * -4);
	\draw (4.5 + 9 * \j, \r * -4) -- (4.5 + 9 * \j, \r * -5);
        \draw (1.5 + 9 * \j, \r * -5) -- (7.5 + 9 * \j, \r * -5);
        
	\foreach \k in {3*\j,3*\j+3,3*\j+6}{
	  \draw (3 * \k + 1.5, \r * -5) -- (3 * \k + 1.5, \r * -6);
	  \draw (3 * \k, \r * -7) -- (3 * \k, \r * -6) -- (3 * \k+3, \r * -6) -- (3 * \k+3, \r * -7) -- (3 * \k, \r * -7);
	  \draw (1.5 + 3 * \k, \r * -7) -- ( 1.5 + 3 * \k, \r * -8);
	  \draw (0.5 + 3 * \k, \r * -8) -- (2.5 + 3 * \k, \r * -8);
	  
	  \foreach \l in {3*\k,3*\k+1,3*\k+2}{
	    \draw (\l + 0.5, \r * -8) -- (\l + 0.5, \r * -9);
	    \draw (\l, \r * -10) -- (\l, \r * -9) -- (\l+1, \r * -9) -- (\l+1, \r * -10) -- (\l, \r * -10); %faudrait faire des cerlces
% 	    \node[draw=none] at (\l + 0.5, \r * -9.5) {id {\l} };

	  }
	  
	  
	}
    }
}
}
\end{tikzpicture}

   }
   \captionof{figure}{HRW with skeleton, $f=2$, $k=3$}
\end{minipage}
\begin{minipage}{0.49\textwidth}
We will use a distributed version of HRW(ref todo ?) with skeleton. Each lvl of the skeleton will correspond to an hyper cube lvl.\\
Nota bene the skeleton has no need to  be construct and store, indeed we only need(ln(|skeleton|) at each request) so we will rebuild a part each time.
\end{minipage}\\

\begin{algorithmic}
    \State v : noeud précédant
    \State u : noeud courant
    \State s : état de l'agent
    \If {$s = 0$}

    \State $i\gets random(0, deg(u)-1)$
    \State $s \gets deg(u)$
    \State $move\_to( end(v,i) )$
       \Else
    \State $p \gets \min_{ \frac{s}{deg(u)}, 1}$
    \If {$random() \leq p$}
      \State $s \gets 0$
    \Else
      \State $reverse\_move$
    \EndIf
      \EndIf
 \end{algorithmic}

% First lets assumes we have the knowledge of the whole map.\\

\subsection{Adding new node}
Problème du *2, inserer par le bas : deplacement de données??

\subsection{Implementation details}
Memoisation of the entire map if network not to huge : $O(1) message$
\section{Placement algorithm}
M is the higest id allocated to a node.
Dead nodes ares stored as a prefix binary tree with the number of leaves stored in each node.
We use a prefix binary tree as a sckeleton for HRW.

dead(prefix) = {nodes in dead nodes with this prefix}
|dead(prefix)| = value of the node indexed by prefix in Dead nodes tree ie (O(log(m-n)~state that m-n is a constant(if m-n>K then reallocate id such that m=n))

LEN = bits len of an identifier\\
len(prefix) = number of bits in prefix\\
$len^{-1} = LEN-len(prefix)$ \\
prefix|X = $prefix * 2 + X$
prefix|0\_ = $prefix * 2^{len^{-1}(prefix)}$\\
prefix|1\_ = $prefix * 2^{len^{-1}(prefix)} + \sum_{i=0}^{len^{-1}(prefix)}{2^i}$\\
load(prefix) = |{number of nodes(alive) with this prefix}| / |{number of nodes(alive)}\\
load(prefix) =  ln{if prefix|1\_ < m then $len^{-1}(prefix)$ else $m-prefix+1$}\\

$max(wrand(prefix|0,x),wrand(prefix|1,x))$\\
$X_i \in [|0..\alpha(i) M|]$, we assume that $\alpha(i) M \in \mathbb{N}$\\
$X_i indep$\\

$\sum_{i=0}^{n}{\beta(i)} = 1$\\
Let $\alpha(i) = \ln{\beta(i)}$\\

\begin{tabular}{lcl}
 $Pr(\max_{i \in [|1..n|]}{X_i} = X_0)$ & = & $Pr(\bigwedge_{k=0}^{\alpha(0) M} X_0=k \wedge \max_{i \in [|1..n|]}{X_i} \le k)$\\
  & $=_{disjoint and indep}$ & $\sum_{k=0}^{\alpha(0) M}{ Pr(X_0=k) \times Pr(\max_{i \in [|1..n|]}{X_i} \le k) }$\\
  %& = & $\sum_{k=0}^{\alpha(i) M}{ \frac{1}{\alpha(0) M + 1} \times Pr(\wegde_{i \in [|1..n|]}{X_i} \le k) }$\\
  & $=_{indep}$ & $\sum_{k=0}^{\alpha(0) M}{\frac{1}{\alpha(0) M + 1} \times \prod_{i=1}^{n}{Pr(X_i \le k) } }$\\
  & $=_{indep}$ & $\sum_{k=0}^{\alpha(0) M}{\frac{1}{\alpha(0) M + 1} \times \prod_{i=1}^{n}{ \frac{k+1}{\alpha(i)M+1} } }$\\
  & $=$ & $\sum_{k=0}^{\alpha(0) M}{(k+1)^{n} \times \prod_{i=0}^{n}{ \frac{1}{\alpha(i)M+1} } }$\\
  & $=$ & $\prod_{i=0}^{n}{ \frac{1}{\alpha(i)M+1} }  \times \sum_{k=0}^{\alpha(0) M}{(k+1)^{n}}$\\
  & $=$ & $\prod_{i=0}^{n}{ \frac{1}{\alpha(i)} } \times \frac{1}{M+1}^{n+1} \times \sum_{k=0}^{\alpha(0) M}{(k+1)^{n}}$\\
  & $=$ & $\prod_{i=0}^{n}{ \frac{1}{\alpha(i)} } \times \frac{1}{M+1} \times (\alpha(0) +\frac{1}{M+1})^{n}$\\
  & $=$ & $\prod_{i=0}^{n}{ \frac{1}{\alpha(i)} } \times (1 +\frac{1}{\alpha(0)(M+1)})^{n}$\\
  & $=$ & $\alpha(0)^{-1} \times \prod_{i=1}^{n}{ \alpha(0)\alpha(i)^{-1} } \times (1 +\frac{1}{\alpha(0)(M+1)})^{n}$\\
  $\ln{}$ & $=$ & $\ln(\alpha(0)^{-1}) + \sum_{i=1}^{n}{ \ln{\alpha(0)} + \ln{\alpha(i)^{-1}} } + \ln{ (1 +\frac{1}{\alpha(0)(M+1)})^{n}}$\\
  & $=$ & $-\beta(0) + n\beta(0) -  \sum_{i=1}^{n}{\beta(i)} + \ln{ (1 +\frac{1}{\alpha(0)(M+1)})^{n}}$\\
  & $=$ & $(n-1)\beta(0) - (1-\beta(0)) + \ln{ (1 +\frac{1}{\alpha(0)(M+1)})^{n}}$\\
  & $=$ & $n\beta(0) - 1 + \ln{ (1 +\frac{1}{\alpha(0)(M+1)})^{n}}$\\
  & $=$ & $n\beta(0) -1 + 1 - \frac{1}{\alpha(0)(M+1)} + o(\frac{1}{\alpha(0)(M+1)}^{2})$\\
  & $=$ & $n\beta(0) - \frac{1}{\alpha(0)(M+1)} + o(\frac{1}{\alpha(0)(M+1)}^{2})$\\
  $\exp{\ln{}}$ & $=$ & $\frac{\exp{n\beta(0)}}{\exp{\alpha(0)(M+1)}} \times \exp{o(\frac{1}{\alpha(0)(M+1)}^{2})}$\\
  & $=$ & $\frac{\exp{n\beta(0)}}{\beta(0)^{M+1}} \times \exp{o(\frac{1}{\alpha(0)(M+1)}^{2})}$\\

  

\end{tabular}

\begin{algorithmic}
    \State Open = $Q_{n}$
    \State Closed =  $\varnothing$
    \State Faces = $\varnothing$
    \While{ $Open \neq \varnothing$ }
      \State $i_{1}, ..., i_{n} = n-min( Open )$
      \For{ $0 < j \leq n$}
	\State $i_{1}, i_{j-1}, k, i_{j+1} i_{n} = n-min( Open )$
      \EndFor
      \State $ $
    \EndWhile
    \If {$s = 0$}

    \State $i\gets random(0, deg(u)-1)$
    \State $s \gets deg(u)$
    \State $move\_to( end(v,i) )$
       \Else
    \State $p \gets \min_{ \frac{s}{deg(u)}, 1}$
    \If {$random() \leq p$}
      \State $s \gets 0$
    \Else
      \State $reverse\_move$
    \EndIf
      \EndIf
 \end{algorithmic}
\end{document}
