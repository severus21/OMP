\section{Metadata management/handling}
\subsection{Hard metadata}
Optional
"Hard metadata" are the metadata needed to rebuild a file from a File ID(it is a Object ID, indeed SHA256(D) ensure unicity)\\

After déduplication we obtains a list of chunks, a 4-uplets list (\textit{pos, length, id, data}).\\

\begin{table}[h]
  \centering
  \begin{tabular}{|c|c|c|}
    \hline
    \textit{id} & ? see object naming & chunk's id, using Object Naming Protocol\ref{}\\
    \textit{pos} & 8 Bytes & Begi"nn"ing of the chunk's data in the original file\\
    \textit{length} & 8 Bytes & Length( bytes) of the chunk's data\\
    \textit{checksum} & 4 Bytes & CRC32-C\\
    \textit{data} & - & chunks' data\\
    \hline
  \end{tabular}
  \caption{}
\end{table}

schema d'explication des niveaux d'indirection
et rapide calcul de la taille max des fichiers stockés

The improvement of the system of hard metadata : it is fully "transparant en terme d'architectur même service que la sauvegarde des chunks".

\textbf{Nota Bene :} To rebuild a file we have only have to use the file retrieving API( chunks number)


\subsection{Soft/abstract metadata}
